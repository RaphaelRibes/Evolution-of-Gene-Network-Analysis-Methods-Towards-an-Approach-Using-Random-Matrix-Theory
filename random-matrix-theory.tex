\subsection{Fundamentals Of Random Matrix}\label{subsec:fundamentals-of-random-matrix}

Random Matrix Theory bridges linear algebra and probability theory\cite{vivo_random_2017}, examining the statistical behavior of matrices with randomly distributed elements.
Originally introduced by Wigner and Dyson in the 1960s to study the spectral properties of complex nuclei, random modeling has since been applied to identifying and analyzing phase transitions linked to disorder and noise\cite{deng_molecular_2012}.
It allows finding order in chaos, revealing underlying structures in complex systems like co-expression networks\cite{luo_constructing_2007}, financial markets\cite{pharasi_complex_2018}, and quantum physics\cite{guhr_random_1997}.
A primary goal of RMT is to study the properties of eigenvalues in matrices with random entries.
\\\\
\noindent TA basic calculation in RMT is finding the spacing distribution of eigenvalues.
A simple example involves a 2x2 real symmetric matrix with Gaussian random variables as entries.

\noindent Consider the matrix $X$\eqref{eq:matrix_X}:
\begin{align}
\label{eq:matrix_X}
X=\begin{pmatrix}
x_1 & x_3\\ x_3 & x_2
\end{pmatrix}\\
\label{eq:N0_1}
x_1, x_2 \sim N(0,1)\\
\label{eq:N0_1/2}
x_3 \sim N(0,\frac{1}{2})
\end{align}


\noindent Where $N(0,1)$\eqref{eq:N0_1} denotes a Gaussian distribution with mean 0 and variance 1 and $N(0,\frac{1}{2})$\eqref{eq:N0_1/2} denotes a Gaussian distribution with mean 0 and variance $\frac{1}{2}$.

The variance of the off-diagonal elements is set to half that of the diagonal elements for a specific reason, enabling an easier analysis.
The question is whether the probability density function (pdf) of the spacing ss between the two eigenvalues can be determined.
So $s=\lambda_1-\lambda_2$ where $\lambda_1$ and $\lambda_2$ are the eigenvalues of the matrix $X$.

\noindent This spacing for a 2x2 matrix can be calculated like this:
\[s = \lambda_1 - \lambda_2 = \sqrt{(x_1 - x_2)^2 + 4x_3^2}\]

\noindent We are going to skip the whole demonstration of the calculation, but the final equation for the pdf of the spacing s is defined like $P(s)=\frac{s}{2}e^{-\frac{s^2}{4}}$.

\begin{figure}[H]
    \captionsetup{aboveskip=5pt, belowskip=5pt} % Adjust spacing here
    \centering
    \includegraphics[width=0.9\textwidth]{Wigner’s surmise} % Path to your image file
    \caption{Wigner’s surmise}
    \label{fig:Wigner’s surmise}
\end{figure}
Despite its simplicity, this result is remarkably profound: it reveals that the probability of sampling two eigenvalues that are "very close" to each other (as $s \to 0$) is extremely low.
It is as though each eigenvalue "senses" the presence of the others and adjusts to maintain a certain distance—neither too close nor too far(\autoref{fig:Wigner’s surmise}).
This behavior is reminiscent of birds perched on an electric wire or cars parked along a street: maintaining a balance between proximity and spacing. \cite{livan_introduction_2017}
\\\\
Random Matrix Theory predicts two universal extreme distributions for the nearest neighbor spacing distribution (NNSD) of eigenvalues: the Gaussian Orthogonal Ensemble (GOE) statistics, reflecting the random characteristics of complex systems, and the Poisson distribution, representing system-specific, nonrandom properties of complex systems.
This transition highlights the level of repulsion, where eigenvalues tend to "avoid" proximity.
The phenomenon underscores intrinsic correlations among eigenvalues, even when matrix entries are independently distributed.

\noindent By investigating eigenvalue spacing distributions, researchers identify key properties like the Wigner's surmise distribution in GOE systems or the exponential decay of Poisson distributions in decoupled systems.
These techniques offer a powerful framework for exploring complex systems, including biological networks and beyond.\cite{luo_application_2006}

The structure of complex systems is better understood through the application of Random Matrix Theory, where eigenvalue spacings are analyzed to reveal transitions from global interactions to modular arrangements.
This approach underscores the utility of statistical models like Wigner’s surmise and Poisson distributions in exploring biological networks and other interconnected systems.

\subsection{Applications Of Random Matrix Theory In Many-Body Systems}\label{subsec:applications-of-random-matrix-theory-in-many-body-systems}

By describing the statistical properties of spectra in complex quantum systems, RMT bridges seemingly disparate phenomena through its universal principles.
The role of RMT in understanding many-body systems, its implications for quantum chaos, and its connections to field theory and statistical mechanics are demonstrated, highlighting its versatility and foundational importance in modern physics.

Many-body systems encompass complex structures involving a lot of particles interacting via two-body forces.
Examples include atomic nuclei, which consist of nucleons bound by strong nuclear forces, and atoms and molecules, where ions and electrons interact through electromagnetic forces.
These systems demonstrate high levels of complexity.
The Hamiltonian is an operator that determines the evolution of a quantum state through the Schrödinger equation.
It is described for $N$ particules like:
\[\hat{H}=\sum^N_{n=1}\hat{T}_n+\hat{V}\]
Where $\hat{T}_n$ is the kinetic energy operator of particle n and $\hat{V}$ is the potential energy function.
The key idea is to replace the complex, specific Hamiltonian of the system with an ensemble of random matrices that share the same symmetries.
\\\\
At low incident energies, the use of the GOE in modeling compound nucleus scattering assumes that the nucleus equilibrates internally faster than it decays.
However, as incident energy increases, the decay time becomes comparable to the equilibration time, meaning the nucleus can decay before full equilibration.

\noindent To address this, the model is extended using the nuclear shell model, dividing the compound nucleus into classes of states with fixed particle-hole numbers.
Each class is represented by a random matrix.
The coupling between neighbors refers to the absence of a fermion in an energy level it would occupy in the ground state, governed by the two-body interaction, is also modeled using random matrices.
Imagine sorting all possible quantum states of a system (like a nucleus) into different groups based on their particle-hole number.
This means each group contains states with the same number of particles excited above the ground state and the same number of holes left behind.
Each of these groups is then modeled using a random matrix.

Instead of trying to calculate the exact energy values for each state in a group, which is extremely slow, complex and laborious, random matrix are used to represent the overall statistical behavior of the energy levels within that group.
The random matrix is chosen from an ensemble that respects the symmetries of the physical system.
The total Hamiltonian is then a band matrix whose entries are random matrices.
By doing this, RMT provides a deeper understanding of resonance behavior and cross-section fluctuations within many-body system models.
