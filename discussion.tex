The evolution of gene network analysis methods, as explored in this work, demonstrates a significant trajectory of progress.
From the foundational applications of graph theory to modern techniques incorporating RMT, this journey reflects the increasing demand for precision, robustness, and interpretability in genomic data analysis.
\\\\
\indent By Integrating RMT, the paper Molecular Ecological Network Analysis represents a great improvement in gene network analysis.
Due to the use of subjective thresholding in traditional methods, it made LCNA less reliable and more prone to biases.
This objective and systematic way to determine thresholds helps to construct networks reliably while minimizing biases.
With this improved accuracy of results and simplicity, MENA has the potential to revolutionize the field of gene network analysis.
\\
\noindent When comparing LCNA to MENA, the respective strengths and limitations are made clear.
LCNA remains a practical entry point for studying simpler gene interactions due to its accessibility and longstanding use in biology.
Because of the simple algorithm to determine the adjacency matrix, MENA stands out as a more objective alternative.
Thanks to this thresholding algorithm, more reliable representation of gene interactions and more accurate identification of hubs in the network.
\\\\
\indent This work has been put the spotlight on the potential of RMT in gene network analysis.
Even if RMT-based techniques address one of the biggest limitations of traditional approaches, they are not without their own challenges.
By default, the simplistic nature of co-expression networks may not capture the full complexity of gene interactions.
Exploring hybrid methodologies is necessary.
If the objectivity of RMT can be combined with the widder range of information used in equation-based models, for example, it could allow more information to be used in the network analysis.
Also, the increase of data available thanks to NGS technologies reinforces the necessity for scalable methods.
\\
\noindent There are a lot of ways to implement more data into co-expression networks.
For example, adding the temporal dynamics that involve the changes in gene expression over time.
This can reveal the regulatory relationships and causal interactions between genes and is crucial for understanding how cells respond to various stimuli.
This additional dimension added to the data is not without its own challenges, as it requires more complex algorithms to analyze.
\\
\noindent Furthermore, MENAP is, for now, only accessible online, making it difficult for researchers to customize or extend its functionalities or just verify the code.
Future developments should focus on creating a user-friendly, open source version of MENAP that allows for greater flexibility and transparency in network analysis.
Broader adoption of the pipeline would be facilitated, and collaboration across research communities would be encouraged, ultimately advancing the understanding of gene networks.
Having a local version of MENAP is expected to make researchers who were not comfortable sending their data to the online version feel more at ease using it.
\\\\
The advancements discussed in this work have implications beyond genomics, extending to fields such as ecology, systems biology, and even financial modeling, where the principles of RMT have found application.
A paradigm shift in the approach to biological data is represented by the integration of RMT into gene network analysis, emphasizing the importance of robustness, objectivity, and scalability in network construction.
The origins of RMT come from an entirely different field.
Its successful application to genomics exemplifying how interdisciplinary research can drive groundbreaking discoveries.
\\
\noindent My hope is that this bibliography serves as a testimony to the power of interdisciplinary collaboration.
The potential for transformative innovation when diverse fields converge is infinite.
And infinity inspired generations of scientists to push the boundaries of what is possible.